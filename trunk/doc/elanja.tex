\documentclass[a4paper,10pt]{article}
\usepackage{amsmath, amsthm, amssymb}
\usepackage[english]{babel}
\usepackage{color,graphicx}

%opening
\title{Intergenerational evolution of attitudes towards integration}
\author{Jacopo Baldassarri, Andrea Blasco, Elisa Omodei}

\begin{document}

\maketitle

\section{The Model}
We consider a population of $N$ agents of two different types: $U = \{a,b\}$.\\
Agents $1,\dots,M$ are of type $a$ and agents $M+1,\dots,N$ of type $b$.\\
Society evolves along $t=0,1,..., T$ interactions.\\
Each agent $i$ has individual preferences towards people types. After t interactions, those are given by the couple of values $(p_t^i, q_t^i)$.\\
The value $p_t^i$ ($q_t^i$) is the probability that a given individual $i$ of type $a$ ($b$) has to meet every other individual of type $a$ ($b$).\\
We say that a link between an individual $i$ and another $j$ is estabilished when both agents have met during the time of one interaction. This happens with a probability $p_t^i p_t^j$ ($q_t^i q_t^j$) when both are of type $a$ ($b$), or with $p_t^i q_t^j$ ($q_t^i p_t^j$) when their types are different.\\
Agent connections have a life-span of one interaction only. After a link has eventually been estabilished, it is used to compute the new individual preferences and then it is removed. This feature should capture the idea of intergenerational transmission of values. So that an agent $i$ at time $t$ can be interpreted as the parent of the same agent $i$ at the following interaction $t+1$.\\
As a consequence of each interaction, individual preferences modify under the following deterministic rule: $p_{t+1}^i$ ($q_{t+1}^i$) equals the fraction of agents of type $a$ ($b$) who are connected at distance $d$ from an agent $i$ at time $t$ over the total number of agents connected at the same distance and time.\\

\end{document}
