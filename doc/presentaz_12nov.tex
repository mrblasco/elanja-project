\documentclass{beamer}
\usetheme{Madrid}

 \mode<presentation>

\usepackage{graphics}
\usepackage[italian]{babel}
\usepackage{default}
\usepackage{verbatim}

\begin{document}
\title{Intergenerational evolution of attitudes towards integration}
\author{Jacopo Baldassarri, Andrea Blasco, Elisa Omodei}
\date{November 12th, 2009}

\frame{
  \titlepage
} 

\frame{ \frametitle{The Model}
\begin{center}
$N$ agents of two different types: $U = \{a,b\}$\\
\uncover<2->{
\

\
Agents $1,\dots,M$ are of type $a$ and agents $M+1,\dots,N$ of type $b$\\}
\uncover<3->{
\

\
Society evolves along $t=0,1,..., T$ interactions}
\end{center}
}

\frame{ \frametitle{The Model}

 Each agent $i$ has individual preferences towards people types. \\After t interactions, those are given by the couple of values $(p_t^i, q_t^i)$.\\
\uncover<2->{
\

\
The value $p_t^i$ ($q_t^i$) is the probability that a given individual $i$ of type $a$ ($b$) has to meet every other individual of type $a$ ($b$).\\}
\uncover<3->{
\

\
We say that a link between an individual $i$ and another $j$ is estabilished when both agents have met during the time of one interaction. \\This happens with a probability $p_t^i p_t^j$ ($q_t^i q_t^j$) when both are of type $a$ ($b$), or with $p_t^i q_t^j$ ($q_t^i p_t^j$) when their types are different.\\}
}

\frame{
 Agent connections have a life-span of one interaction only. \\After a link has eventually been estabilished, it is used to compute the new individual preferences and then it is removed. \\This feature should capture the idea of intergenerational transmission of values. \\So that an agent $i$ at time $t$ can be interpreted as the parent of the same agent $i$ at the following interaction $t+1$.\\
\uncover<2->{
\

\
As a consequence of each interaction, individual preferences modify under the following deterministic rule: \\$p_{t+1}^i$ ($q_{t+1}^i$) equals the fraction of agents of type $a$ ($b$) who are connected at distance $d$ from an agent $i$ at time $t$ over the total number of agents connected at the same distance and time.\\}
}

\frame{ 
We define:\\
\begin{center}
\begin{verbatim}
double p[COLUMN];
double q[COLUMN];
double R[ROW][COLUMN];
int A[ROW][COLUMN];
int degree[ROW];
int A_degree[ROW], B_degree[ROW];
\end{verbatim}
\end{center}
}

\frame{
\begin{center}
At $t=0$ we use a random Adiajency Matrix and we compute 
\begin{verbatim}
 q[i] = A_degree[i] / degree[i]
 p[i] = B_degree[i] / degree[i]
\end{verbatim}
\end{center}
}

\frame{
We create a symmetric random matrix $NxN$ with $0$ on the diagonal (called R) and we compute the Adiajency Matrix at $t=1$ in this way:\\
\scriptsize
\begin{verbatim}
/* Adjacency Matrix */ 
       for(i=0; i<ROW; i++)
       {
	for(j=i; j<COLUMN; j++)
	{
   		if(j < M && i < M ) 
		{	
			link = (double) *(R+(i*COLUMN)+j) - *(p+i) * *(p+j);
		} else if (j >= M  && i < M ) 
		{
			link = (double) *(R+(i*COLUMN)+j) - *(q+i)* *(p+j);
			
		} else if (j >= M && i >= M) 
		{
			link = (double) *(R+(i*COLUMN)+j) - *(q+i) * *(q+j);	
		}
		if (link < 0) 
		{
			*(A+(i*COLUMN)+j) = 1; 		 
		}	 
	}
  }
\end{verbatim}
}

\frame{
 We simulate a society composed of $100$ individuals: $50$ of type $a$ and $50$ of type $b$.\\
\uncover<2->{
\

\
If in the computation of the probabilities we consider as friends only the agents at a distance $d=1$, than after N interactions we'll have a situation of \textit{Segregation}.}
\uncover<3->{
\

\
This happens because when the probability of meeting an agent of the other type becomes zero it will stay zero indefinitely.}
}

\frame{
\begin{center}
$t=2$
%\includegraphics[width=0.6\textwidth]{step2.png}
\end{center}
}

\frame{{The first Simulation}
\begin{center}
$t=3$
%\includegraphics[width=0.6\textwidth]{step3.png}
\end{center}
}

\frame{{The first Simulation}
\begin{center}
$t=100$
%\includegraphics[width=0.6\textwidth]{step100.png}
\end{center}
}

\frame{{The next step}
 The next step is to see what happens if we consider as friends also the agents linked at a distance $d=2$.\\
\uncover<2->{
\

\
This case shouldn't lead to segregation. \\Infact when a probability becomes zero, it can become nonzero during the following steps.}
}

\frame{{An example}
 Let's consider the following random Adiajency Matrix at $t=0$
\begin{displaymath}
A = \left( \begin{array}{cccc} 1 & 0 & 0 & 0 \\ 0 & 1 & 1 & 0 \\ 0 & 1 & 1 & 1 \\ 0 & 0 & 1 & 1 \end{array} \right)
\end{displaymath}
The probabilities at $d=2$ are:
\begin{displaymath}
 \begin{array}{cc} p_1=1 & q_1=0 \\ p_2=\frac{1}{3} & q_2=\frac{2}{3} \\ p_3=\frac{1}{3} & q_3=\frac{2}{3} \\ p_4=\frac{1}{3} & q_4=\frac{2}{3} \end{array}
\end{displaymath}
}

\frame{{An example}
 At $t=1$ we can get
\begin{displaymath}
A = \left( \begin{array}{cccc} 1 & 1 & 0 & 0 \\ 1 & 1 & 0 & 1 \\ 0 & 0 & 1 & 1 \\ 0 & 1 & 1 & 1 \end{array} \right)
\end{displaymath}
The probabilities at $d=2$ are:
\begin{displaymath}
 \begin{array}{cc} p_1=\frac{2}{3} & q_1=\frac{1}{3} \\ p_2=\frac{1}{2} & q_2=\frac{1}{2} \\ p_3=\frac{1}{3} & q_3=\frac{2}{3} \\ p_4=\frac{1}{2} & q_4=\frac{1}{2} \end{array}
\end{displaymath}
\\$q_1$ isn't zero anymore!
}


\end{document}

