\documentclass[12pt,titlepage]{article}
\usepackage{amsmath, amsthm, amssymb}
\usepackage[english]{babel}
 \usepackage{color,graphicx}
%\usepackage{subfig}
\newcommand{\Andr}[1]{\textcolor{blue}{#1}}
\newcommand{\Elisa}[1]{\textcolor{red}{#1}}
%\newcommand{\informatico}[1]{\textcolor{orange}{#1}}

%xy-pic
\usepackage[all]{xy}

%opening
\title{Brain-storming}
\author{Corso Di fisica ... Rambaldi... }


\begin{document}
\maketitle
\Andr{ Lo so che sono stato poco chiaro ma di Brain-storming si trattava... cmq per ogni modifica o chiarimento che vuoi aggiungere a questo documento puoi usare il comando in $Tex$ $slash + $ Elisa$\{ ... \} $, il testo che inserisci diventa "rosso"  cosi'  ci sappiamo distinguere ...}


\section*{Il/ Un Modello/ino}
Supponiamo di voler studiare il comportamento di una collettivita' di persone classificabili in uno di due distinti "tipi" (ad esempio l'etnia). Queste interagiscono tra di loro seguendo una regola di comportamento che e' uguale per tutti (invariante rispetto al tipo) e che tuttavia puo' portare la societa' a convergere verso modelli sociali d'integrazione oppure di divisione tra le persone di tipi differenti.\\

Per modellizzare questa situazione si rappresenta la societa' come una collezione di due insiemi non-vuoti $G(A,E)$. L'insieme degli agenti $A=\{1,2,...,N\}$ e quello delle relazioni che li legano tra loro $E$; per esempio esiste una relazione tra $n,m\in A$  se "$n$ $e'$ $amico$ $di$ $m$".\\ 
Per semplicita' si assume soltanto un' unica relazione possibile,
%\footnote{ Ogni relazione puo' essere interpretata come un "link" bidirezionale  in un grafo $G(A,E)$ in cui i gli agenti siano "nodi".}, 
ogni elemento $e_{n,m} \in E$ indica se essa esiste ed e' biunivoca ($e_{n,m}=1$) oppure non c'e' ($e_{n,m}=0$).
\begin{figure} 
\centering
\xymatrix{
  &&&& {\bullet}^n \ar[rr]_{e_{n,m}}  & & {\bullet}^m \ar[ll] 
  } 
 \caption{\label{links} La relazione esistente tra due agenti.}
\end{figure}

A prescindere dalle proprie relazioni, ogni individuo appartiene ad uno tra due distinti "gruppi sociali" o "etnie" o "tipi" ecc. .\\ 
%L'appartenenza si denota come $\omega_n \in \{0,1\}$.\\
La societa' si modifica attraverso un numero indefinito di periodi $t=0,1,...$. L'insieme di agenti resta immutato (cosi' come il tipo), ma le relazioni cambiano. Al periodo iniziale $t=0$, ciascuno ha la possibilita' di formare una relazione o link con $ogni$ altro agente. L'evento accade con una probabilita' $P_0 < 1$ che e' uguale per tutti e che determina un numero atteso di connessioni pari ad $(N-1) P_0$ per ciascuno.\\
Alla fine del periodo, e' possibile caratterizzare la societa' come relativamente "integrata" oppure "divisa". Una misura di questo fenomeno e' data dalla $composizione$ $media$ $dei$ $tipi$ che appartengono al "vicinato" di ogni agente \footnote{Per vicinato (dall inglese $neighborhood$) di un agente si intede l'insieme degli agenti con cui e' stato stabilito un link diretto.}. \\
Nel periodo immediatamente successivo $t=1$ , i link creati vengono rimossi e si genera una nuova possibilita' di stabilire relazioni.\\ 
Adesso la probabilita' di collegarsi agli altri non e' piu' quella iniziale, ma si e' modificata in base alle passate relazioni (come se venisse ereditata l'esperienza da padre in figlio). Il modo in cui questo avviene e' stabilito da una funzione che associa a ciascun tipo e  per una data composizione del vicinato una nuova probabilita' $P_t^n$. Ne consegue che il valore atteso di connessioni non sara' piu' lo stesso per tutti.   In oltre e' interessante pensare al caso in cui, invece di avere una singola probabilita' di connessione, gli agenti hanno diversi incentivi ad incontrare persone a seconda del loro tipo. Per esempio se denotiamo il tipo con la variabile $\omega_n$, avremo che un agente $n$ si collega ad $m$ con la seguente probabilita': 
\[
	P_t^n(\omega_m;\omega_n) = p_t^n \cdot p_t^m \hspace{1cm} \text{if} \;\; \omega_m=\omega_n
\]
\[
	P_t^n(\omega_m;\omega_n) = q_t^n \cdot q_t^m\hspace{1cm} \text{if} \;\; \omega_m \neq \omega_n
\]  
Quando $p_t^n > q_t^n$ diremo che c'e' $homophily$ (preferenza per il proprio tipo) o viceversa che c'e' $heterophily$.\\
Dopo un certo numero di interazioni e' possibile guardare a come la societa' si e' evoluta, se integrata oppure divisa.

\subsection*{Problemi}
Il modello puo' esssere considerato come una semplificazione di una societa' in cui gli agenti nascono e muoiono, in cui le generazioni si alternano perfettamente. Tuttavia si assume che i social network passati  (ovvero tutte le connessioni occorse) non hanno effetti diretti sulle connessioni delle generazioni future,  ma influiscono attraverso le modifiche alle probabilita'.\\
Il brutto di questa cosa e' che ci impone di utilizzare e quindi inventare di sana pianta una funzione generatrice delle probabilita'.  Un modo migliore di affrontare il problema potrebbe essere quello di pensare a come la struttura della rete di relazioni possa determinare gli stessi effetti di divisione od unita' sociali. Per esempio, in un contesto di immigrazione le connessioni di chi nasce da genitori immigrati sono legate al tessuto di relazioni dei propri parenti (erano integrati?). Hanno poco a che vedere con le volonta' individuale di stare con persone di un certo tipo.

\subsection*{Candidata "funzione generatrice"}
Guardando alla figure \ref{update}, abbiamo un nodo $n$ con quattro connessioni o relazioni; due di queste relazioni sono intrattenute con persone di un tipo differente (cerchio colorato), le altre due sono intrattenute con persone dello stesso tipo (cerchio). Possiamo pensare ad una regola adattiva per questo nodo: con probabilita' $3/5$ si vorra' collegare a persone della suo tipo mentre con $2/5$ si collega agli altri ($5$ e' il numero di agenti del suo vicinato -- agente incluso). 
 \begin{figure} 
\centering
\xymatrix{
&&&  & {\circ}  \ar[dr]  & {\bullet} \ar[d]  &&\\
 &&& &   & {\circ}^n \ar[u] \ar[d] \ar[dr] \ar[ul]  & &    \\
 &&& &   & {\bullet} \ar[u]  & {\circ} \ar [ul] &  
  } 
 \caption{\label{update} I cerchi rappresentano un tipo mentre i cerchi neri sono l'altro tipo.}
\end{figure}


 
\end{document}
